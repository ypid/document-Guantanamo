\section{Schließung}
Die Schließung des Gefängnislagers wurde schon oft gefordert zum Beispiel von der
UN-Menschenrechtskommission (Februar 2006).
Schon George W. Bush, ehemaliger Präsident der Vereinigten Staaten, kündigte mehrmals die Schließung
an. Die er allerdings nie umsetzte.
Bush sagte einst: \enquote{Wir sind in einem Krieg gegen diese Terroristen.
Es ist meine Aufgabe und Pflicht, die amerikanische Bevölkerung vor weiteren Terrorattacken
zu schützen}\footcite{DW:Guantanamo}.
In diesem Punkt muss ich ihm allerdings widersprechen, auch wenn mir klar ist, dass von ihm
als Präsident so etwas erwartet wird.
Meiner Meinung nach ist die wahre Gefahr, die von Terrorismus ausgeht, nicht die Gefahr, dass eine Person
Opfer von Terroranschlägen wird, sondern vielmehr, wie die Politik und Gesellschaft auf Terrorismus
reagiert. So lassen wir uns freiwillig immer weiter Überwachen und Kontorollieren, nur um eine,
jetzt schon, sehr kleine Wahrscheinlichkeit weiter zu minimieren. Da hierdurch jegliche
Freiheit und Individualität verloren gehen könnte.
Wodurch auch die Kreativität der Menschen
abnehmen wird.

Guantanamo Bay ist hier ein eindrückliches Beispiel wie Grenzen überschritten werden, wenn man
versucht, Terrorismus um jeden Preis zu verhindern. Dies könnte auch ein heimliches Ziel von
(schlauen) Terroristen sein.

Barack Obama, jetziger Präsident der USA, führte schon seinen Wahlkampf mit dem Versprechen,
innerhalb eines Jahres, Guantanamo zu schließen.
%% http://www.focus.de/politik/ausland/uswahl/guantanamo-obama-ordnet-schliessung-an_aid_364103.html
Nun sind schon über \CalcAge[yearsuffix=false]{2009}{01}{22} vergangen und noch immer ist
Guantanamo Bay in Betrieb. Dies wird so erklärt, dass es schwierig ist, Staaten zu finden, die
Guantanamo-Gefangene aufnehmen.
Es wird teils sogar berichtet, dass die Situation mit Obama für die Gefangenen schlimmer
geworden sei.\footcite{AlJazeera:Obama_bad}

Diese Probleme sollte Obama versuchen zu lösen, wenn er nicht in die Fußstapfen seines Vorgängers
treten will.\footnote{Abbildung \url{http://cache.boston.com/universal/site_graphics/blogs/bigpicture/44_01_21/4442_17682555.jpg}}

\begin{comment}
\begin{figurewrapper}
	\includegraphics[width=0.9\hsize]{files/images/4442_17682555}
	\captionof{figure}[Obama in den Fußstapfen von Bush]{Obama in den Fußstapfen von Bush?
		\newline Quelle:
\url{http://cache.boston.com/universal/site_graphics/blogs/bigpicture/44_01_21/4442_17682555.jpg}}
\end{figurewrapper}
\end{comment}
