\section{Menschenrechte wertlos -- rechtsfreier Raum Guantanamo}
Die USA pachtet seit 1903, für eine unbefristete Zeit, ein \SI{117}{\square\kilo\metre} großes,
zuvor durch die USA besetztes, Gebiet auf Kuba. Hier betreibt die US Navy die Guantanamo Bay Naval
Base. Abgesegnet von George W. Bush wurden seit 2002 mehrere Gefangenenlager in diesem Stützpunkt
errichtet. Zurzeit sind noch circa 171 Personen\footcite{Reuters:Obama} inhaftiert.

Gefangen genommen werden Personen, denen unterstellt wird, eine Gefahr für die USA zu sein. Die
Häftlinge werden verhört und gefoltert, um sie zu einem Geständnis zu bewegen. Die angewendeten
Methoden verstoßen nicht nur gegen Menschenrechte, sondern auch gegen die Verfassung der Vereinigten
Staaten.

Hierbei setzen die Verantwortlichen eine Vielzahl von Techniken ein.
Die alle samt sehr unangenehm und die meisten davon illegal sind.
In der folgenden Liste führe ich einige dieser Techniken auf.
\begin{itemize}
	\item Schlafentzug durch lange Verhöre, in denen ein Häftling angekettet auf einem Stuhl sitzt.
	\item Psychische Folter durch Todesdrohungen, Beschimpfungen, sexuelle Erniedrigung oder durch
		Ausnutzung der Angst vor Hunden.
	\item Aufhängen der Häftlinge an Handschellen oder Anketten, sodass sie sich kaum noch bewegen
		können.
	\item \enquote{Waterboarding} bei dem der Eindruck des Ertrinkens erzeugt wird.
\end{itemize}

Diese Behandlung führt oft zu einer starken Depression und zu Persönlichkeitsveränderungen.
Es gibt auch mehrerer Fälle von Selbstmord unter den Gefangenen.
Das könnte auch ein Grund für die USA sein die, teils unschuldigen, Häftlinge nicht freizulassen.
Denn \enquote{wenn sie nicht schon zuvor Terroristen waren, dann sind sie es jetzt
sicherlich.}\footnote{Zitat des US-Militärs aus \cite{Telepolis:Guantanamo}}
Die USA erschafft sich ihre Feinde sozusagen selbst.

Die Regierung der USA benutzt zudem, meiner Meinung nach, verschiedene Gesetzeslücken, um die
Aktivitäten in Guantanamo Bay legal aussehen zu lassen. Außerdem werden Informationen,
die das Gegenteil beweisen, oft gezielt zurückgehalten.
\begin{itemize}
	\item Die Gefangenenlager sind nicht innerhalb der USA, somit ist die zivile Gerichtsbarkeit
		quasi unmöglich.\footnoteremember{\cite{DW:Guantanamo}}{cite:DW:Guantanamo}
	\item Die Häftlinge werden nicht mal als \enquote{Kriegsgefangene} anerkannt -- als solche
		hätten sie bestimmte Rechte -- sondern als \enquote{feindliche
		Kämpfer}.\footnoterecall{cite:DW:Guantanamo} %So gelten sie quasi als Vogelfrei
\end{itemize}

Für die Häftlinge gibt es keinen Gerichtsprozess und meist haben sie auch keinen Kontakt zur Außenwelt,
noch nicht einmal zu einem Anwalt. Je nach Grad der \enquote{Kooperationsbereitschaft} werden die
Häftlinge in verschiedene Klassen eingeteilt. Diese Klassen kann man an der Häftlingsuniform
erkennen. Die Kooperationsbereiten tragen weiße Kleidung alle andern orangefarbene.

Vereinzelt werden auch Gefangene freigelassen. Für diese wird es aber sehr schwierig, sich wieder an
ein normales Leben zu gewöhnen und sich in die Gesellschaft zu integrieren.

%Wie kann es sein, dass in Guantanamo Bay über \CalcAge{2002}{01}{01} Folter betrieben wird und
%Menschenrechte verletzt werden?
